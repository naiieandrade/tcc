\chapter{Introdução}


O acesso à internet transformou a forma de buscar conhecimento e se relacionar em sociedade. A quantidade de informações encontradas na internet gerou o maior acervo de todos os tempos com conteúdos em todos os países, incluindo textos, imagens e vídeos. E com a evolução dos meios de comunicação em massa, se antes já foi preciso dias até a veiculação de notícias ou informações, hoje elas são disponibilizadas em menos de segundos devido à tecnologia.

Com o aumento contínuo de pessoas usando essas tecnologias, os fornecedores de serviços se depararam com um enorme volume de clientes. Com o desafio de  melhorar essa experiência individualmente e mantê-los conectados dentro dos parâmetros que geram lucro máximo, essas empresas investem na personalização de serviços, 
%utilizando os dados como insumo e tendo como auxílio o uso de algoritmos e matemática, visando o melhor atendimento para cada usuário.
utilizando os dados como insumo nos algoritmos de personalização. A melhora de experiência viria com a competição, que obrigaria empresas a inovarem o produto para se manterem vivas e lucrativas.

%Desde então dados têm se tornado o novo minério de ouro e é impossível não notar sua obtenção por meio de formulários, redes sociais, aplicativos, compras, entre outros.  Muitos deles são obtidos por meio de dispositivos com acesso à internet, como celulares. Assim, empresas conseguem obter informações de localização, tipo de dispositivo, o tempo que utilizam e quais serviços e dados pessoais.

O anúncio da Google representou um marco nessa revolução importante, porém quase invisível, no modo como são consumidas as informações. Segundo Pariser (\citeyear{pariser2012}), em dezembro de 2009, começou a era da personalização. Para ele, a internet iria democratizar o planeta, conectando pessoas e informações e traria uma espécie de utopia global libertadora. Entretanto, seguindo essa lógica, os algoritmos se tornaram os curadores da entrega de resultados e as interações passaram a serem mediadas pelos interesses das empresas que fazem a curadoria do conteúdo da internet. %por meio de filtros, aumentando o tempo de permanência de um usuário na rede e fazendo com que criadores de conteúdo invistam em conteúdo relevante dentro da rede social para conseguir a atenção das pessoas.

%Empresas como Google e o Facebook são exemplos que evidenciam os diferentes filtros aplicados de forma particularizada, levando em consideração a quantidade de usuários que possuem em todo o mundo. XXXXXX 

A ampla maioria das pessoas imagina que os mecanismos de busca sejam imparciais. Mas essa percepção talvez se deva ao fato de que esses mecanismos são cada vez mais parciais, adequando-se a visão de mundo de cada um. Gradativamente, o monitor do computador se torna uma espécie de espelho que reflete os próprios interesses de cada um, baseando-se na análise de cliques feita por observadores algorítmicos \cite{pariser2012}.

Para o criador da Web, Berners (\citeyear{berners2017}), a Web seria como uma plataforma aberta que permitiria que todos compartilhassem informações.  %Entretando, E mesmo nao sendo a mesma web..
%Entretanto a Web que Mesmo estando à altura dessa visão, existe uma batalha recorrente para mantê-la aberta.
Entretanto, a Web que existe hoje não é a mesma que foi imaginada na sua concepção e aponta preocupação com três tendências que comprometem o seu verdadeiro potencial 
%Ele também aponta preocupação com três tendências que comprometem o seu verdadeiro potencial
%: atender a humanidade.
como uma ferramenta que serve toda a humanidade.

A primeira preocupação é o controle que as empresas detêm sobre os dados pessoais e o modelo de negócio utilizado em muitos sites que oferecem conteúdo gratuito em troca desses dados. Por meio de termos longos e confusos, são retirados das pessoas o controle sobre seus dados, não dando a liberdade aos usuários escolher quais dados ou com qual empresa compartilhar. 
%Estes, que são mantidos por silos proprietários que retiram o controle de direito sobre esses dados. Além da falta de privacidade, onde a leis favorecem cada vez mais os (desfavorecem/atropelam) os direitos à privacidade. %Não dando a liberdade aos usuários escolher quais dados ou com qual empresa compartilhar. onde os usuários se tornam refém das
%A primeira preocupação é o controle que as empresas têm sobre os dados pessoais, que é o modelo de negócio de muitos sites que oferecem conteúdo gratuito em troca desses dados, onde os usuários se tornam refém das .%, que ficam fora do conhecimento dos fornecedores....%, que ficam "sujeitos" 
%E nesse processo as pessoas não podem deicidir com quem compartilhar ou quais dados são, já que estão em silo proprietário..
%Estes dados ficam fora da visão dos usuários, 
%O controle sobre os dados pessoais 
%O modelo de negócios de muitos sites  
A segunda é a facilidade que desinformações se espalham na Web na busca de notícias e informações, % principalmente em sites de mída social e mecanismos de pesquisa, por
%visto que a busca por notícias e informações em sua maioria são encontradas em sites de mídia social e mecanismos de pesquisa.
que em sua maioria são encontradas em sites de mídias sociais e mecanismos de pesquisa. Com base 
%em algoritmos que aprendem com os
nos dados pessoais que são constantemente coletados, os resultados são mostrados com base nos conteúdos mais prováveis de serem lidos e não pela relevância do conteúdo. Os sites ganham de acordo a quantidade de cliques e se a tecnologia for má utilizada pode haver manipulação para ganhos financeiros ou políticos. 
%com a tecnologia má utilizada pode espalhar desinformações com fins financeiros ou políticos. %ganhos
% são mostrados conteúdo de relevância
%Com má intenção podem espalhar desinformação por ganhos financeiros ou políticos.
A terceira é a falta de transparência da publicidade política, que com uma rica base de dados pessoais resultam na criação de anúncios individuais direcionados diretamente aos usuários. A propaganda direcionada permite que uma campanha propague informações diferentes para grupos diferentes, desta forma, agindo de maneira antiética e não clara. 

%sendo utilizada de maneira antiética e contrariando a democracia.
%é resultado da
%e ele aponta preocupação a três tendências: controle sobre os dados pessoais
%A facilidade da desinformação 
%Transparência e compreensão na publicidade política online
%Em 2014... \cite{staltz2017}

As gigantes da tecnologia Google, Facebook e Amazon são especialistas em seus serviços oferecidos e também são exemplos de empresas que geram as preocupações levantadas por Berners (\citeyear{berners2017}). Assim, Staltz (\citeyear{staltz2017}) explana como a dinâmica do poder na Web tem mudado drasticamente com essas três empresas principais no centro dessa transformação. %da web.
Aproximadamente três quartos dos acessos dos principais provedores de conteúdo 
%do conteúdo da internet é indicado 
são indicados por uma das duas plataformas: Google e Facebook.
%Atualmente, Google e Facebook têm influência direta em mais de 70\% de todo o tráfego da internet,
%mesmo que tenham praticamente os mesmos recursos e interface de usuários antes mesmo do ano de 2014. 
Após essas empresas terem tentado competir com serviços similares entre si e falharem %, enquanto ainda existia opção e alguma diversidade,
 elas se especializaram, diminuindo a diversidade no mercado da internet, as opções dos usuários, buscando lideranças de mercado e eliminando concorrência.
 %Enquanto essas empresas ainda competiam mantinham a diversidade, 
 %no mercado da internet.
%entretanto com as empresas dominantes Google, Facebook e Amazon, a diversidade diminui

Todos esses modelos de negócio e comportamentos da sociedade influenciam no surgimento dos filtros bolhas que são as informações que os algoritmos direcionam a pessoas com perfil de interesse parecido. Isso gera no usuário uma sensação de estar cercado de pessoas de opiniões parecidas, distanciando-o assim de “bolhas” diferentes, informações diferentes e pessoas diferentes, bloqueando conhecimentos e evitando discussões e resolução de conflitos. Assim, o cenário atual cria desafios para a participação social na internet já que, ao invés de trabalhar para o fortalecimento das bolhas de opinião, as plataformas de participação devem devem tentar efetivamente combatê-las.
%Esse cenário cria

\section{Justificativa}

Este trabalho faz parte do desenvolvimento da plataforma Empurrando Juntos (EJ), que foi idealizado e desenvolvido pelo Laboratório Avançado de Produção Pesquisa e Inovação em Software (LAPPIS), da Universidade de Brasília em conjunto com o Instituto Cidade Democrática, em comum acordo com o Hacklab. 

Empurrando Juntos é uma plataforma que organiza tópicos de discussão em torno de "conversas". As conversas, que podem ser criadas por qualquer pessoa, definem uma temática que permitem a criação de comentários que os participantes podem concordar, discordar ou pular. 
%se abster
Desta forma, com a participação dos usuários gradativamente é possível reconhecer os diferentes perfis que agrupam pessoas semelhantes, também chamados grupos de opiniões, que serão disponibilizados para todos os participantes. 
%Esses perfis agrupam pessoas semelhantes 
% Esses perfis são grupos de pessoas que são similares a partir dos votos. 
A fim de entregar transparência e controle dos dados para cada usuário, com a opção de extrair métricas para orientar ou justificar decisões e comparar opiniões próprias com o todo \cite{mendes2019}.

%de serviços que possibilita a criação de conversas e que por meio da utilização de gamificação identifica perfis específicos. Existem diferentes tipos de perfis de utilização, e possibilita poderes temporários com o objetivo de manter a diversidade nos debates.


%As conversas são criadas por qualquer pessoa e as demais a recebem em sua conta por meio de notificação push, com três opções disponíveis, sendo elas concordar com a conversa; discordar da conversa; e pular a conversa. Também é permitido comentários, assim como concordar ou não com comentários já feitos.
%Essas são as informações de entrada que são utilizadas para a definição de perfis, além de informações opcionais de registro pessoal, como sexo e idade. 

%A arquitetura se resume em um aplicativo e uma plataforma de serviços em Software Livre que se conecta com aplicativos \textit{crowdsource} de participação e utiliza as notificações push para potencializar aspectos como o debate informado, a diversidade de opinião e a ação coletiva.

Este trabalho tem o objetivo de classificar os perfis de usuários, com algoritmos que aproximem pessoas com pensamentos próximos, e que facilite em futuro a visualização desses dados de forma simples e clara para os usuários. A transparência dessas informações é de grande importância para que as pessoas possam se identificar nas suas bolhas e ter a compreensão do todo na tentativa de manter debates saudáveis.
A plataforma também evidencia a presença de bolhas de opinião quando elas estiverem presentes e propõe mecanismos para diminuir estas bolhas quando elas ocorrerem.

\section{Objetivos}

\subsection{Objetivo Geral}

Este trabalho tem o propósito de avaliar os sistemas de classificação de perfis de usuários da plataforma Empurrando Juntos e indicar métricas de performance e confiabilidade.
%assim como maneiras de transparecer a melhor forma de visualização das bolhas de opinião para os usuários do Empurrando Juntos.

%Este trabalho tem o propósito de avaliar os sistemas de classificação de perfis de usuários da plataforma e indicar métricas de performance e confiabilidade, assim como maneiras de transparecer a melhor forma de visualização das bolhas de opinião para os usuários do Empurrando Juntos.

\subsection{Objetivos Específicos}

\begin{itemize}
	%\item Visualização de dados e classificação de perfil em uma plataforma de participação social;
	%\item Realizar estudo técnico sobre estrutura de dados;
	\item Realizar estudo técnico sobre algoritmos de detecção de perfis de opinião;
	\item Avaliar e otimizar os algoritmos de detecção de perfis de opinião;
	\item Analisar e aplicar algoritmos de clusterização nos dados reais da plataforma;
	%\item Avaliar a confiabilidade dos modelos e indicar métricas de confiabilidade;
	\item Propor modelos estatísticos para os dados e realizar testes com dados sintéticos;
%	\item Explorar técnicas de visualização de dados e grupos de opinião. %para potencializar o debate informado
	%\item Explorar modelos de visualização de bolhas;
	%\item Realizar estudo técnico comparativo das visualizações existentes;
	%\item Formalizar e analisar resultados.
	
\end{itemize}

\section{Metodologia de Trabalho}


%Para atingir o objetivo deste trabalho, 

O desenvolvimento deste trabalho se baseia na elaboração de dados sintéticos mais próximos possíveis da realidade para que os algoritmos de classificação e visualização possam ser avaliados e na análise de dados reais utilizados no EJ. Para o sucesso do objetivo, serão estudados e colocados em prática os modelos estatísticos conhecidos, assim como a compreensão de comportamentos e padrões dos usuários. A implementação faz uso da linguagem de programação Python, ferramentas e bibliotecas de ciência de dados e aprendizado de máquina. E para as visualizações dos grupos de opinião, serão analisadas as diversas formas e qual atende ao propósito do EJ. 

Todo o trabalho será realizado por meio de uma plataforma de versionamento de código. Mantendo todas as versões já passadas, o estado atual e a evolução de todo o estudo.

\section{Organização do Trabalho}

Este trabalho está organizado em 5 capítulos. O \autoref{chap:2} apresenta o levantamento feito de plataformas que apresentam visualizações de grupos de opinião. No \autoref{chap:3} é apresentado o estudo sobre modelos estatísticos necessários para o entendimento e alcance do objetivo deste trabalho. O \autoref{chap:4} traz a metodologia utilizada e os resultados obtidos. O \autoref{chap:5} relata os resultados alcançados e no \autoref{chap:conclusao} a conclusão do trabalho.

