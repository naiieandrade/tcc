\chapter{Introdução}


O acesso à internet transformou principalmente a forma de buscar conhecimento e se relacionar. A quantidade de informações encontradas na internet gerou o maior acervo de todos os tempos com conteúdos de todos os países, inúmeros textos, imagens e vídeos. E com a evolução dos meios de comunicação em massa, se antes já foi preciso dias até a veiculação de notícias ou informações, hoje são disponibilizadas em menos de segundos devido à tecnologia.

Com o aumento contínuo de pessoas usando essas tecnologias, os fornecedores de serviços se depararam com um enorme volume de clientes. Com o desafio de  melhorar essa experiência e mantê-los conectados, essas empresas investem na personalização de serviços, 
%utilizando os dados como insumo e tendo como auxílio o uso de algoritmos e matemática, visando o melhor atendimento para cada usuário.
utilizando os dados como insumo nos algoritmos de personalização, visando o melhor atendimento para cada usuário.

%Desde então dados têm se tornado o novo minério de ouro e é impossível não notar sua obtenção por meio de formulários, redes sociais, aplicativos, compras, entre outros.  Muitos deles são obtidos por meio de dispositivos com acesso à internet, como celulares. Assim, empresas conseguem obter informações de localização, tipo de dispositivo, o tempo que utilizam e quais serviços e dados pessoais.

O anúncio do Google representou um marco nessa revolução importante, porém quase invisível, no modo como são consumidas as informações. Segundo \cite{pariser2012}, em dezembro de 2009, começou a era da personalização. Para ele a internet iria democratizar o planeta, conectando informações e traria uma espécie de utopia global libertadora. Entretanto, os algoritmos se tornaram os curadores da entrega de resultados seguindo essa personalização, por meio de filtros, aumentando o tempo de permanência de um usuário na rede e fazendo com que criadores de conteúdo invistam em conteúdo relevante dentro da rede social para conseguir a atenção das pessoas.

%Empresas como Google e o Facebook são exemplos que evidenciam os diferentes filtros aplicados de forma particularizada, levando em consideração a quantidade de usuários que possuem em todo o mundo. XXXXXX 

A ampla maioria das pessoas imagina que os mecanismos de busca sejam imparciais. Mas essa percepção talvez se deva ao fato de que esses mecanismos são cada vez mais parciais, adequando-se a visão de mundo de cada um. Cada vez mais, o monitor do computador é um espécie de espelho que reflete os próprios interesses de cada um, baseando-se na análise de cliques feita por observadores algorítmicos \cite{pariser2012}.

Para o criador da Web, \cite{berners2017}, a Web seria como uma plataforma aberta que permitiria que todos compartilhassem informações.  %Entretando, E mesmo nao sendo a mesma web..
%Entretanto a Web que Mesmo estando à altura dessa visão, existe uma batalha recorrente para mantê-la aberta.
Entretanto, a Web que existe hoje não é a mesma que  \cite{berners2017} imaginou na sua criação e aponta preocupação com três tendências que comprometem o seu verdadeiro potencial 
%Ele também aponta preocupação com três tendências que comprometem o seu verdadeiro potencial
%: atender a humanidade.
como uma ferramenta que serve toda a humanidade.

A primeira preocupação é o controle que as empresas têm sobre os dados pessoais e o modelo de negócio utilizado em muitos sites que oferecem conteúdo gratuito em troca desses dados. %Não dando a liberdade aos usuários escolher quais dados ou com qual empresa compartilhar. onde os usuários se tornam refém das
%A primeira preocupação é o controle que as empresas têm sobre os dados pessoais, que é o modelo de negócio de muitos sites que oferecem conteúdo gratuito em troca desses dados, onde os usuários se tornam refém das .%, que ficam fora do conhecimento dos fornecedores....%, que ficam "sujeitos" 
%E nesse processo as pessoas não podem deicidir com quem compartilhar ou quais dados são, já que estão em silo proprietário..
%Estes dados ficam fora da visão dos usuários, 
%O controle sobre os dados pessoais 
%O modelo de negócios de muitos sites  
A segunda é a facilidade que desinformações se espalham na Web na busca de notícias e informações,% principalmente em sites de mída social e mecanismos de pesquisa, por
%visto que a busca por notícias e informações em sua maioria são encontradas em sites de mídia social e mecanismos de pesquisa.
que em sua maioria são encontradas em sites de mídia social e mecanismos de pesquisa. Com base 
%em algoritmos que aprendem com os
nos dados pessoais que são constantemente coletados, os resultados são mostrados por relevância, ou seja, conteúdos mais prováveis de serem lidos.% são mostrados conteúdo de relevância
%Com má intenção podem espalhar desinformação por ganhos financeiros ou políticos.
A terceira é a falta de transparência da publicidade política, que com uma rica base de dados pessoais resultam na criação de anúncios individuais direcionados diretamente aos usuários.

%é resultado da
%e ele aponta preocupação a três tendências: controle sobre os dados pessoais
%A facilidade da desinformação 
%Transparência e compreensão na publicidade política online
%Em 2014... \cite{staltz2017}

As gigantes da tecnologia Google, Facebook e Amazon são especialistas em seus serviços oferecidos e também são exemplos de empresas que geram as preocupações levantadas por  \cite{berners2017}. Assim, \cite{staltz2017} explana como a dinâmica do poder na Web tem mudado drasticamente com essas três empresas principais no centro dessa transformação. %da web.
Aproximadamente três quartos do conteúdo da internet é indicado por uma das duas plataformas: Google e Facebook.
%Atualmente, Google e Facebook têm influência direta em mais de 70\% de todo o tráfego da internet,
%mesmo que tenham praticamente os mesmos recursos e interface de usuários antes mesmo do ano de 2014. 
Após essas empresas terem tentado competir com serviços similares entre si e falharem%, enquanto ainda existia opção e alguma diversidade,
 elas se especializaram no fazem de melhor diminuindo a diversidade no mercado da internet, as opções dos usuários, buscando lideranças de mercado e eliminando concorrência.
 %Enquanto essas empresas ainda competiam mantinham a diversidade, 
 %no mercado da internet.
%entretanto com as empresas dominantes Google, Facebook e Amazon, a diversidade diminui

Todos esses comportamentos da sociedade influenciam no surgimento dos filtros bolhas que são as informações que os algoritmos direcionam a pessoas com perfil de interesse parecido. Isso gera no usuário uma sensação de estar cercado de pessoas de opiniões parecidas, distanciando-o assim de “bolhas” diferentes, informações diferentes e pessoas diferentes, bloqueando conhecimentos e evitando discussões. Assim, o cenário atual cria desafios para a participação social na internet já que, ao invés de trabalhar para o fortalecimento das bolhas de opinião, as plataformas de participação devem devem tentar efetivamente combatê-las.
%Esse cenário cria

\section{Justificativa}

Este trabalho faz parte do desenvolvimento da plataforma Empurrando Juntos (EJ), que foi idealizado e desenvolvido inicialmente pelo Laboratório Avançado de Produção Pesquisa e Inovação em Software (LAPPIS), da Universidade de Brasília em conjunto com o Instituto Cidade Democrática, em comum acordo com o Hacklab. 

Empurrando Juntos é uma plataforma que organiza tópicos de discussão em torno de "conversas". As conversas, que podem ser criadas por qualquer pessoa, definem uma temática que permitem a criação de comentários que os participantes podem concordar, discordar ou pular. 
%se abster
Desta forma, com a participação dos usuários gradativamente é possível reconhecer perfis com opiniões diferentes, os chamados grupos de opiniões, que serão disponibilizados para todos os participantes. Entregando valor de transparência e controle dos dados para cada usuário, podendo extrair métricas para orientar ou justificar decisões e comparar opiniões próprias com o todo \cite{mendes2019}.

%de serviços que possibilita a criação de conversas e que por meio da utilização de gamificação identifica perfis específicos. Existem diferentes tipos de perfis de utilização, e possibilita poderes temporários com o objetivo de manter a diversidade nos debates.


%As conversas são criadas por qualquer pessoa e as demais a recebem em sua conta por meio de notificação push, com três opções disponíveis, sendo elas concordar com a conversa; discordar da conversa; e pular a conversa. Também é permitido comentários, assim como concordar ou não com comentários já feitos.
%Essas são as informações de entrada que são utilizadas para a definição de perfis, além de informações opcionais de registro pessoal, como sexo e idade. 

%A arquitetura se resume em um aplicativo e uma plataforma de serviços em Software Livre que se conecta com aplicativos \textit{crowdsource} de participação e utiliza as notificações push para potencializar aspectos como o debate informado, a diversidade de opinião e a ação coletiva.

Este trabalho tem o objetivo de classificar os perfis de usuários, com algoritmos que aproximem pessoas com pensamentos próximos, e tenha uma visualização desses dados de forma simples e clara para os usuários. A transparência dessas informações é de grande importância para que as pessoas possam se identificar nas suas bolhas e ter a compreensão do todo e manter debates saudáveis.
que evidenciem a presença de bolhas de opinião quando elas estiverem presente.

\section{Objetivos}

\subsection{Objetivo Geral}

Este trabalho tem o propósito de classificar os perfis de usuários da plataforma e transparecer a melhor forma de visualização das bolhas de opinião para os usuários do Empurrando Juntos, a fim de fomentar debates e discussões.

\subsection{Objetivos Específicos}

\begin{itemize}
	\item Visualização de dados e classificação de perfil em uma plataforma de participação social;
	%\item Realizar estudo técnico sobre estrutura de dados;
	\item Realizar estudo técnico sobre algoritmos de detecção de perfis de opinião;
	\item Avaliar e otimizar os algoritmos de detecção de perfis de opinião;
	\item Realizar testes com dados sintéticos;
	\item Explorar técnicas de visualização de grupos de opinião. %para potencializar o debate informado
	%\item Explorar modelos de visualização de bolhas;
	%\item Realizar estudo técnico comparativo das visualizações existentes;
	%\item Formalizar e analisar resultados.
	
\end{itemize}

\section{Metodologia de Trabalho}

Para o desenvolvimento deste trabalho será necessário mockar dados mais próximos possíveis da realidade para que a visualização atenda aos dados reais.

Para isso será necessário:
\begin{itemize}
	\item A criação da base da dados por distribuição estatística: distribuição Dirichlet.
	\item Análise de modelos: Naive Bayes, LDA.
	\item Análise de visualizações
\end{itemize}

\section{Organização do Trabalho}

Este trabalho está organizado em 3 capítulos. O Capítulo 2 apresenta 

