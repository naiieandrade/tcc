\chapter{Introdução}

O avanço das tecnologias de informação e comunicação (TIC) vêm mudando a forma de viver das pessoas. As TICs têm tomado diversas áreas das atividades humanas \cite{monereo2010} e vêm moldando e marcando a sociedade da informação. O uso de dispositivos, dos mais diversos tamanhos, surpreendem com suas funcionalidades auxiliando nas tarefas do dia-a-dia. \cite{martins2005} afirma que distintas práticas sociais estão cada vez mais orientadas por e para essas tecnologias.

As TICs também são reconhecidas como agentes de mudança no setor público e como instrumentos que viabilizam a implementação de processos inovadores na gestão. Se tornando cada vez mais necessária a adaptação do governo ao ambiente digital que tem ganhado espaço entre cidadãos e empresas. Exemplos desses benefícios são as possibilidades de melhorar a comunicação entre governo e cidadãos com mecanismos de participação democrática, colaboração na definição de políticas públicas, entrega de serviço e acesso aos dados públicos. \cite{tic2015}

Já o acesso a internet transformou principalmente a forma de buscar conhecimento e se relacionar. A quantidade de informações encontradas na internet gerou o que se considera, atualmente, o maior acervo de todos os tempos com conteúdo de todos os países, inúmeros textos, imagens e vídeos. E com a evolução dos meios de comunicação em massa, se antes já foi preciso dias até a veiculação de notícias ou informações, hoje são disponibilizadas em menos de segundos devido à tecnologia.

Ainda no século XX uma nova economia surgiu em escala global. \apud{castells1999}{neves2000} a chama de informacional, global e em rede. É informacional porque a produtividade e a competitividade dependem da capacidade de gerar, processar e aplicar de forma eficiente a informação baseada em conhecimentos. E desde então a criação e obtenção de dados nos últimos anos supera todos os anos anteriores desde o início da nossa história.

Desde então dados têm se tornado o novo minério de ouro e é impossível não notar sua obtenção por meio de formulários, redes sociais, aplicativos, compras, entre outros.  Muitos deles são obtidos por meio de dispositivos com acesso à internet, como celulares. Assim, empresas conseguem obter informações de localização, tipo de dispositivo, o tempo que utilizam e quais serviços e dados pessoais.

Com o aumento contínuo de pessoas usando essas tecnologias, os fornecedores de serviços se depararam com um enorme volume de clientes. Com o desafio de  melhorar essa experiência e mantê-los conectados, essas empresas investem na personalização de serviços,  tendo como auxílio o uso de algoritmos e matemática, visando o melhor atendimento para cada usuário.

Serviços como Google e o Facebook são exemplos que evidenciam os diferentes filtros aplicados de forma particularizada, levando em consideração a quantidade de usuários que possuem em todo o mundo. E o uso de algoritmos personalizados, por exemplo, a pesquisa de “células troncos” pode gerar resultados diferentes para um ambientalista e para um executivo. A ampla maioria das pessoas imagina que os mecanismos de busca sejam imparciais. Mas essa percepção talvez se deva ao fato de que esses mecanismos são cada vez mais parciais, adequando-se a visão de mundo de cada um. Cada vez mais, o monitor do computador é um espécie de espelho que reflete os próprios interesses de cada um, baseando-se na análise de cliques feita por observadores algorítmicos. \cite{pariser2012}

O anúncio do Google representou um marco em uma revolução importante, porém quase invisível, no modo como são consumidas as informações. Segundo \cite{pariser2012}, em dezembro de 2009, começou a era da personalização. Para ele a internet iria democratizar o planeta, conectando informações e traria uma espécie de utopia global libertadora. Entretanto, os algoritmos se tornaram os curadores da entrega de resultados seguindo essa personalização, por meio de filtros, aumentando o tempo de permanência de um usuário na rede e fazendo com que criadores de conteúdo invistam em conteúdo relevante dentro da rede social para conseguir a atenção das pessoas.

Todos esses comportamentos da sociedade influenciam no surgimento dos filtros bolhas que são as informações que os algoritmos direcionam a pessoas com perfil de interesse parecido. Isso gera no usuário uma sensação de estar cercado de pessoas de opiniões parecidas, distanciando-o assim de “bolhas” diferentes, informações diferentes e pessoas diferentes, bloqueando conhecimentos e evitando discussões.

\section{Justificativa}

Este trabalho faz parte do desenvolvimento da plataforma “Empurrando Juntos”, que foi idealizado e desenvolvido inicialmente pelo Laboratório Avançado de Produção Pesquisa e Inovação em Software (LAPPIS), da Universidade de Brasília em conjunto com o Instituto Cidade Democrática, em comum acordo com o Hacklab. 
Empurrando Juntos é uma plataforma de serviços que possibilita a criação de conversas, e que por meio da utilização de gamificação identifica perfis específicos. Existem diferentes tipos de perfis de utilização, e possibilita poderes temporários com o objetivo de manter a diversidade nos debates.

As conversas são criadas por qualquer pessoa e as demais a recebem em sua conta por meio de notificação push, com três opções disponíveis, sendo elas concordar com a conversa; discordar da conversa; e pular a conversa. Também é permitido comentários, assim como concordar ou não com comentários já feitos. Essas são as informações de entrada que são utilizadas para a definição de perfis, além de informações opcionais de registro pessoal, como sexo e idade. 

A arquitetura se resume em um aplicativo e uma plataforma de serviços em Software Livre que se conecta com aplicativos crowdsource de participação e utiliza as notificações push para potencializar aspectos como o debate informado, a diversidade de opinião e a ação coletiva.

Este trabalho tem o objetivo de criar os perfis de usuários, com algoritmos que aproximem pessoas com pensamentos próximos, e tenha uma visualização desses dados de forma simples e clara para os usuários. A transparência dessas informações é de grande importância para que as pessoas possam se identificar nas suas bolhas e ter a compreensão do todo e manter debates saudáveis.

\section{Objetivos}

\subsection{Objetivo Geral}

Este trabalho tem o propósito de classificar os perfis de usuários da plataforma e transparecer a melhor forma de visualização das bolhas de opinião para os usuários do Empurrando Juntos, a fim de fomentar debates e discussões.

\subsection{Objetivos Específicos}

\begin{itemize}
	\item Visualização de dados e classificação de perfil em uma plataforma de participação social;
	\item Realizar estudo técnico sobre estrutura de dados;
	\item Realizar estudo técnico sobre algoritmos de personalização;
	\item Explorar modelos de visualização de bolhas;
	\item Realizar estudo técnico comparativo das visualizações existentes;
	\item Formalizar e analisar resultados.
\end{itemize}

\section{Metodologia de Trabalho}

Para o desenvolvimento deste trabalho será necessário mockar dados mais próximos possíveis da realidade para que a visualização atenda aos dados reais.

Para isso será necessário:
\begin{itemize}
	\item A criação da base da dados por distribuição estatística: distribuição Dirichlet.
	\item Análise de modelos: Naive Bayes, LDA.
	\item Análise de visualizações
\end{itemize}

\section{Organização do Trabalho}

Este trabalho está organizado como ...

