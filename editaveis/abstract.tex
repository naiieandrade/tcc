\begin{resumo}[Abstract]
 \begin{otherlanguage*}{english}

The large volume of users on the Internet has encouraged companies to create business models that consume personal data to build more accurate user profiles. Customization algorithms and relevancy content choices put people in opinion bubbles and use this information opaquely in their business models, even compromising the freedom of the Web itself. And with this challenge, Empurrando Juntos (EJ) is a participatory platform that seeks to deliver transparency, knowledge of data to each user and prevent the dynamics of bubble formation from influencing users' opinions. The aim of this paper is to evaluate EJ user profiles and identify the best rating ways that make it easier for users to view opinion bubbles in order to foster debate and discussion. The approach uses synthetic data and some real data based on known statistical models, so that they represent the actual data reliably.


   \vspace{\onelineskip}
 
   \noindent 
   \textbf{Key-words}: Opinion bubbles. Social participation. Preview Profile classification.
 \end{otherlanguage*}
\end{resumo}
