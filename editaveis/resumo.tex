\begin{resumo}

O grande volume de usuários na internet incentivou as empresas a criarem modelos de negócio que consomem dados pessoais para construir perfis mais precisos de seus usuários. Algoritmos de personalização e escolhas de conteúdos por relevância, colocam as pessoas em bolhas de opinião e utilizam essa informação de maneira opaca em seus modelos de negócio, comprometendo até mesmo a liberdade da própria Web. E com esse desafio, o Empurrando Juntos (EJ) é uma plataforma participativa que busca entregar transparência, conhecimento dos dados para cada usuário e evitar que a dinâmica de formação de bolhas influenciem as opiniões dos usuários. O objetivo deste trabalho é avaliar os perfis de usuários do EJ e identificar as melhores formas de classificação que facilte a visualização das bolhas de opinião para os usuários, a fim de fomentar debates e discussões. A abordagem utiliza dados sintéticos e alguns dados reais com base em modelos estatísticos conhecidos, de maneira que representem de forma fidedigna os dados reais.


 \vspace{\onelineskip}
    
 \noindent
 \textbf{Palavras-chaves}: Bolhas de opinião. Participação social. Visualização. Classificação de perfil.
\end{resumo}
