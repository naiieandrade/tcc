\chapter{Visualização de dados de plataformas de participação}
\label{chap:2}
\section{Pol.is}

Pol.is é uma das referências deste trabalho e foi utilizado como uma das referências também de Software Livre para a construção do Empurrando Juntos por abordar as dimensões de governança digital, inclusão e manipulação \cite{poppi2017}. É uma plataforma busca entregar transparência a organizações e seus membros, e que os diversos pontos de vista possam ser reconhecidos em uma conversa \footnote{\href{https://pol.is/home}{https://pol.is/home}}. 
%que ajuda as organizações a se entenderem visualizando o que seus membros pensam . Para que tenham uma visão clara de todos os pontos de vista para ajudar a levar a conversa adiante.
A inteligência do Pol.is foi utilizada nas primeiras versões do EJ.

Os criadores do Pol.is apontam a ineficiência na comunicação de grandes grupos de pessoas sobre determinados tópicos como o problema motivador para a criação da plataforma. Então, buscaram combinar técnicas de aprendizado de máquina e visualização interativa de dados em tempo real para Web. Afirma que o visual é voltado para o usuário, de forma simples e limpa, buscando estimular conversas e engajar participantes \footnote{\href{https://www.geekwire.com/2014/startup-spotlight-polis/}{https://www.geekwire.com/2014/startup-spotlight-polis/}}.

%Ser a primeira empresa a combinar o aprendizado de máquina online (redes neurais..) com visualização interativa de dados em um aplicativo de Web em tempo real.

%Para os criadores do Pol.is, o problema era simples: a ineficiência de  grandes grupos de pessoas tentando se comunicar efetivamente sobre um determinado tópico on-line. E por isso desenvolveram uma maneira de combinar dados de pesquisa de centenas de pessoas com aprendizado de máquina e visualização interativa de dados. O resultado final é uma maneira simples e limpa para qualquer um, enquanto permite que os usuários estimulem conversas com base em todas as informações. \footnote{https://www.geekwire.com/2014/startup-spotlight-polis/}

O objetivo dos criados do Pol.is sempre foi mostrar os grupos de opinião. E no processo de concepção dessa forma de visualização, no início queriam mostrar a “distância” entre os participantes com base em padrões de votação semelhantes e diferentes na conversa como elos, e esse agrupamento emergiria naturalmente disso, entretanto não foi isso que aconteceu. A primeira tentativa foi a visualização por meio de rede de grafos, entretanto perceberam que tinha muitas informações à mostra, e decidiram não mostrar algumas informações utilizando a matemática. \footnote{\href{https://blog.pol.is/the-evolution-of-the-pol-is-user-interface-9b7dccf54b2f}{https://blog.pol.is/the-evolution-of-the-pol-is-user-interface-9b7dccf54b2f}}

%\subsection{Evolução da visualização do Pol.is}




%\begin{figure}[h]
%	\centering
%	\includegraphics[keepaspectratio=true,scale=0.4]{figuras/evolucao-polis-1.png}
%	\caption{Análise de grafos do Pol.is}
%	Fonte: \url{https://blog.pol.is/the-evolution-of-the-pol-is-user-interface-9b7dccf54b2f}
%	\label{fig01}
%\end{figure}

A utilização de Análise de Componentes Principais (ou PCA em inglês), o que hoje é cerne do Pol.is mostra os dois primeiros principais componentes Fig. \ref{fig02} nos eixos x e y. Ocorre a perda de alguns dados na compactação para duas dimensões, mas preserva as maiores diferenças de opinião. 

Na Fig. \ref{fig02} é possível observar que cada participante é mostrado como um ponto preto e os vetores do PCA são expostos na visualização. Clicar nos círculos nos eixos traria os comentários associados àquele vetor.% - pessoas mais à esquerda, por exemplo, teriam maior probabilidade de concordar com alguma coleção de comentários.


\begin{figure}[h]
	\centering
	\includegraphics[keepaspectratio=true,scale=0.2]{figuras/evolucao-polis-2.png}
	\caption{Participantes e vetores do PCA do Pol.is}
	Fonte: \url{https://blog.pol.is/the-evolution-of-the-pol-is-user-interface-9b7dccf54b2f}
	\label{fig02}
\end{figure}


Pol.is utilizou \textit{k-means} aos pontos e eliminou os pontos que tinham menos de um certo número de votos (eles tendiam a se agrupar no centro). Isso melhorou a sensação e começou a transmitir a ideia principal - há grupos de participantes que votaram de maneira semelhante e são um grupo porque compartilham um certo número de perspectivas, não apenas uma.

%Dividiu os usuários em forma de seta, dimensionado proporcionalmente e um marcador no mapa, círculo azul, para enfatizar o aspecto espacial. Seguido da tarefa de criar uma correlação mais forte entre o comentário selecionado e o estado da visualização.

A suposição levantada pelos criados sobre o anonimato era muito restritiva. E colocar as pessoas na visualização resolveria todos os tipos de problemas, inclusive tornando a visualização muito mais concreta. O resultado deste trabalho pode ser visto na Fig. \ref{fig03}. este foi o resultado 


\begin{figure}[h]
	\centering
	\includegraphics[keepaspectratio=true,scale=0.3]{figuras/evolucao-polis-3.png}
	\caption{Plataforma do Pol.is}
	Fonte: \url{https://blog.pol.is/the-evolution-of-the-pol-is-user-interface-9b7dccf54b2f}
	\label{fig03}
\end{figure}



\section{ConsiderIt}

ConsiderIt foi criado na Universidade de Washington, como parte da pesquisa de doutorado financiada pela National Science Foundation, com o objetivo de criar um método pelo qual grandes grupos de pessoas pudessem deliberar juntos e encontrar um terreno comum, mesmo em tópicos controversos 
\footnote{\href{https://consider.it/tour?feature=moderation\#research}{https://consider.it/tour?feature=moderation\#research}}.

A plataforma ConsiderIt por meio de sua interface traz em sua abordagem questionamentos onde os usuários podem criar ou votar em comentários já existentes, que são dividos em prós e contras. E por meio dessas interfaces esperam facilitar, engajar os usuários e trazer reflexão sobre as diversas perspectivas \cite{bennett2012}.


ConsiderIt foi construído a partir do básico da deliberação pessoal para promover uma deliberação pública mais eficaz. É focado em fazer as pessoas pensarem sobre as compensações de uma ação proposta, como uma medida em uma eleição, convidando-os a criar uma lista de prós e contras como mostra a Fig. \ref{fig04}. Em vez de apenas ter a opção binária de concorda ou não, existe a possibilidade de proporcionalidade de opinião, e a criação das listas com prós e contras. 

\begin{figure}[h]
	\centering
	\includegraphics[keepaspectratio=true,scale=0.35]{figuras/considerit-tema2.png}
	\caption{ConsiderIt}
	Fonte: \url{https://consider.it/examples}
	\label{fig04}
\end{figure}


ConsiderIt reaproveita essas deliberações pessoais para oferecer um guia em evolução para o pensamento público e apresenta as considerações mais notáveis pró e contra baseadas na frequência com que são incluídas e se são incluídas por pessoas com diferentes posições sobre o assunto. Também permite aprofundar os pontos relevantes para diferentes segmentos da população, podendo assim gerar \textit{insights} sobre as considerações de pessoas com diferentes perspectivas, podendo ajudar os usuários a identificar áreas comuns inesperadas. 

Também contribui com uma métrica de classificação de pró/contra feita para destacar pontos que ressoam com um público diverso, para promover pontos persuasivos e, ao mesmo tempo, incentivar uma diversidade de pontos de vista e, com sorte, resistir à manipulação estratégica.


\section{PolitEcho}

PolitEcho mostra o enviesamento político de amigos do Facebook e \textit{feed} de notícias de um usuário. É uma extensão do Google Chrome que conecta com o Facebook e atribui a cada amigo uma pontuação baseada em uma previsão de tendências políticas e exibe um gráfico da lista de amigos. Em seguida, calcula o viés político no conteúdo do feed de notícias e compara-o com o viés da lista de amigos para destacar possíveis diferenças entre os dois. As cores azul e vermelho representam viés liberal e conservador respectivamente como pode ser visto na Fig. \ref{fig05}.


\begin{figure}[h]
	\centering
	\includegraphics[keepaspectratio=true,scale=0.3]{figuras/politecho.png}
	\caption{PolitEcho}
	Fonte: \url{https://politecho.org/}
	\label{fig05}
\end{figure}


As avaliações políticas dos amigos são baseadas nas páginas políticas do Facebook que eles gostam. As páginas que os amigos gostaram são comparadas em um banco de dados de páginas do Facebook que foram classificadas por seu viés liberal/conservador e, é computado uma pontuação com base em quaisquer correspondências \footnote{\href{https://politecho.org/}{https://politecho.org/}}.\