\begin{agradecimentos}

%Quando a gente entra na faculdade o momento mais esperado é o fim

%Quando eu inicio uma nova etapa da minha vida as expectativas sempre são altas quando imagino aquele 
%sempre me pego imaginando aquela cena do 
%objetivo sendo realizado para então sentir o tão sonhado dever cumprido. 



%Ao fim de mais este ciclo eu fico muito feliz por hoje olhar pra trás e me orgu

% Mas é no processo que eu aprendi a valorizar cada apoio, cada palavra, cada energia boa e cada tudo. 

Ao fim de mais este ciclo eu fico muito feliz por tudo o que precisei passar e por todas as pessoas que foram colocadas na minha vida para estar saindo uma pessoa diferente da que entrou na UnB-FGA em 2012. Foram longos anos, muitos momentos de altos e baixos que eu superei. E por isso meu primeiro agradecimento é para Deus, porque sem Ele eu não teria forças e nem pessoas verdadeiras comigo nessa caminhada.


% muitas pessoas nessa caminhada. E por isso meu primeiro agradecimento é pra Deus, que colocou todas essas pessoas na minha vida e os obstáculos em que eu já chorei bastante, mas superei.

%Foram longos anos na FGA, muitos momentos altos e baixos, muitas pessoas nessa caminhada. E por isso meu primeiro agradecimento é pra Deus, que colocou todas essas pessoas na minha vida e os obstáculos em que eu já chorei bastante, mas superei.

E agradeço a toda minha família só por ser minha, porque eu os amo e essa foi a força que precisei quando estava triste. Sempre com muito orgulho de mim sempre foi minha base. Minha mãe e pai nunca me deixaram faltar nada, me deram até demais. E tudo para que eu concluísse meu objetivo, mesmo que não entendam até hoje meus sonhos eles confiam em mim. Obrigada Maria das Neves e Reginaldo. Agradeço toda a minha família, meus primos queridos, minhas tias e tios, minha cunhada, meu cunhado e aos meus irmãos Aninha e André. 

Um agradecimento especial ao meu padrinho Maurício e ao meu primo Vitor que foram inspirações minhas para que eu cursasse Engenharia. Meu primo Kevin que sempre soube me dar conselhos quando eu estava para baixo. 


%Obrigada mãe pelas viagens, mesmo que no fim do semestre, era o que salvava. E agradeço a toda minha família só por ser minha, porque eu os amo e essa foi a força que precisei quando estava triste.

Agradeço todos os meus amigos por toda essa caminhada universitária juntos. Quem passou por essa experiência sabe que não é fácil. E vocês todos foram minha rede de apoio, por me escutarem, me motivarem, acreditarem em mim e sempre estarem lá quando eu precisei. Também não tivemos só momentos difíceis, nos divertimos bastante. Obrigada de coração aos amigos de longa data Ananda, Bruno, Carol, Jéssyca, Juliana, Marianna e aos que eu ganhei na UnB, especialmente o Matheus Figueiredo. 

A faculdade também me deu meu namorado e amigo, Matheus Batista, que eu agradeço imensamente por sua sinceridade e por acreditar em mim nos momentos em que eu não acreditei. Obrigada por sempre me ajudar e aprender comigo, e por me ensinar a ser uma pessoa melhor todos os dias. 

%por atenderem minhas ligações chorando, por me dar conselhos, por me incentivarem e motivarem e sempre estarem lá quando eu precisei. Vocês são o máximo! Agradeço também o meu namorado e amigo por sua sinceridade e por acreditar em mim nos momentos em que eu não acreditei, e por me ajudar e aprender comigo.

Por fim, um agradecimento aos professores da UnB, especialmente aos que tive aula e por quem tenho carinho especial até hoje, mas não vou citar nomes porque a lista é extensa. Se estou saindo da faculdade diferente de quando entrei é por conta de todos os aprendizados e exemplos que tive na faculdade desses professores incríveis. Obrigada ao Profº. Dr. Fábio Macedo Mendes e a Profª. Dra. Marília Miranda Gomes por toparem serem meus orientadores e ter tido essa oportunidade incrível de aprender muito mais com vocês!

%um pouquinho mais com vocês!

%que hoje também fazem parte de mim. Obrigada Fábio e Marília por toparem serem meus orientadores e ter tido essa oportunidade incrível de aprender um pouquinho mais com vocês! 






%e \textit{espaço simples e fonte padrão do texto (sem negritos, aspas ou itálico}.

%\textbf{Caso não deseje utilizar os agradecimentos, deixar toda este arquivo em branco}.

\end{agradecimentos}
